% based on http://www.dfcd.net/articles/latex/latex.html
% modified version of https://github.com/xtotdam/latex-templates/blob/master/header.tex

\documentclass[a4paper,11pt]{article}
\usepackage{cmap}               % seems to enable search in pdf
\usepackage[T1,T2A]{fontenc}    % for correct work with cyrillic letters
\usepackage[utf8]{inputenc}
\usepackage{amssymb,amsfonts}
\usepackage[fleqn]{amsmath}
\usepackage[russian,english]{babel}
\usepackage{anyfontsize}
\usepackage{graphicx}
\usepackage{mathrsfs}           % calligraphic letters for math environment
\usepackage[usenames,dvipsnames,svgnames,table]{xcolor}

\usepackage{geometry}   % page geometry: margins
\geometry{left=2cm}
\geometry{right=1.5cm}
\geometry{top=2cm}
\geometry{bottom=2cm}

\usepackage{fancyhdr}   % must be loaded after geometry!
\setlength{\headheight}{15.2pt}
\renewcommand{\headrulewidth}{0pt}
\pagestyle{fancy}

\DeclareMathOperator{\sinc}{sinc}
\DeclareMathOperator{\const}{const}
\DeclareMathOperator{\Tr}{Tr}   %trace
\DeclareMathOperator{\Dim}{dim} %dimensions

\let\vaccent=\v % rename builtin command \v{} to \vaccent{}
\renewcommand{\v}[1]{\ensuremath{\mathbf{#1}}} % for vectors
\newcommand{\gv}[1]{\ensuremath{\mbox{\boldmath$ #1 $}}} % for vectors of Greek letters
\newcommand{\uv}[1]{\ensuremath{\mathbf{\hat{#1}}}} % for unit vector

\newcommand{\abs}[1]{\left| #1 \right|} % for absolute value
\newcommand{\norm}[1]{\lVert #1 \rVert} % for norm ||f||
\newcommand{\qnorm}[1]{\lVert #1 \rVert ^2} % for norm quadrat ||f||^2
\newcommand{\avg}[1]{\left< #1 \right>} % for average

\newcommand{\vmult}[2]{[\vec{#1}{\times}\vec{#2}]} % vector multiplication
\newcommand{\scmult}[2]{(\vec{#1}{\cdot}\vec{#2})} % scalar multiplication

\let\underdot=\d % rename builtin command \d{} to \underdot{}
\renewcommand{\d}[2]{\frac{d #1}{d #2}} % for derivatives
\newcommand{\dd}[2]{\frac{d^2 #1}{d #2^2}} % for double derivatives
\newcommand{\pd}[2]{\frac{\partial #1}{\partial #2}} % for partial derivatives
\newcommand{\pdd}[2]{\frac{\partial^2 #1}{\partial #2^2}} % for double partial derivatives
\newcommand{\pddv}[3]{\frac{\partial^2 #1}{\partial #2 \partial #3}} % for double partial derivatives, diff. variables
\newcommand{\pdc}[3]{\left( \frac{\partial #1}{\partial #2} \right)_{#3}} % for thermodynamic partial derivatives

\newcommand{\ket}[1]{\left| #1 \right>} % for Dirac kets
\newcommand{\bra}[1]{\left< #1 \right|} % for Dirac bras
\newcommand{\braket}[2]{\left< #1 \vphantom{#2} \right|\left. \!#2 \vphantom{#1} \right>} % for Dirac brackets
\newcommand{\matrixel}[3]{\left< #1 \vphantom{#2#3} \right| #2 \left| #3 \vphantom{#1#2} \right>} % for Dirac matrix elements

\newcommand{\comml}[2]{\left[ \hat{#1}, #2 \right]} % [A,...]
\newcommand{\commr}[2]{\left[ #1, \hat{#2} \right]} % [...,B]
\newcommand{\comm}[2]{\left[ \hat{#1}, \hat{#2} \right]} % for commutator for only two operators like [A,B]
\newcommand{\commi}[4]{\left[ \hat{#1}_{#2}, \hat{#3}_{#4} \right]} % for commutator for only two operators like [A,B] with indices

\let\divsymb=\div % rename builtin command \div to \divsymb
\renewcommand{\div}[1]{\operatorname{div}\vec{#1}} % dirty hack for divergence
\newcommand{\rot}[1]{\operatorname{rot}\vec{#1}}
\newcommand{\grad}[1]{\operatorname{grad}\vec{#1}}

\newcommand{\gradn}[1]{\vec{\nabla} #1} % for gradient
\newcommand{\divn}[1]{\scmult{\nabla}{#1}} % for divergence
\newcommand{\rotn}[1]{\vmult{\nabla}{#1}} % for curl

\newcommand{\CCBYSA}[0]{\textcircled{\scalebox{.5}{CC}} \textcircled{\scalebox{.5}{BY}} \textcircled{\scalebox{.5}{SA}}}
\newcommand{\MIT}[0]{MIT License}

\usepackage{scrtime}
\newcommand{\praiseme}[1]{\scalebox{.4}{\MIT,~\copyright~#1, compiled \the\day.\the\month.\the\year\textunderscore\thistime}}

\newcommand{\tobewritten}[0]{\textcolor{red}{\textsc{To be written}}}
\newcommand{\NB}[0]{\textbf{NB!}}
\newcommand{\Def}[0]{$\mathfrak{Def.}$}
\newcommand{\Th}[0]{$\mathfrak{[Th.]}$}
\newcommand{\Wiki}[0]{$\mathfrak{(Wiki)}$}

\newcommand{\bbar}[1]{\bar{\bar{#1}}}

\newcommand{\negphantom}[1]{\settowidth{\dimen0}{#1}\hspace*{-\dimen0}}

\usepackage[inline]{enumitem}
\makeatletter
\newcommand{\inlineitem}[1][]{
    \ifnum\enit@type=\tw@
        {\descriptionlabel{#1}}
        \hspace{\labelsep}
    \else
        \ifnum\enit@type=\z@
        \refstepcounter{\@listctr}\fi
    \quad\@itemlabel\hspace{\labelsep}%
    \fi
}
\makeatother


\usepackage{indentfirst}
\usepackage{float}
\usepackage{longtable}
\usepackage{hyperref}
\usepackage{array}
\usepackage{cancel}
\newcolumntype{x}[1]{>{\centering\arraybackslash\hspace{0pt}}m{#1}}

\usepackage{titlesec}
\titleformat{\chapter}[display]
{\normalfont\huge\bfseries}{\chaptertitlename\ \thechapter}{20pt}{\Huge}
\titlespacing*{\chapter}{0pt}{-50pt}{40pt}

\begin{document}
\setlength{\tabcolsep}{8pt}
\renewcommand{\arraystretch}{1.2}

\tableofcontents
% \newpage

\vspace{3cm}
\begin{center}
\begin{tabular}{r r l}
Acronym & & Compound \\\hline
MCA  & CAA  & chloroacetic acid \\
DCA  & DCAA & dichloroacetic acid \\
TCA  & TCAA & trichloroacetic acid \\
MFA  & FAA  & fluoroacetic acid \\
DFA  & DFAA & difluoroacetic acid \\
TFA  & TFAA & trifluoroacetic acid \\
CPA  & 2CPA & 2-chloropropionic acid \\
FPA  & 2FPA & 2-fluoropropionic acid \\
~    & DFPA & difluoropropionic acid \\
~    & TFPA & tetrafluoropropionic acid \\
% &    2Pr-OH & 2-propanol \\
\end{tabular}
\end{center}

% THIS IS TO BE DELETED OF COURSE
\vspace{1cm}
{\huge\bf Beware! The data below is totally random! Thus nan's and regression f*ckups are totally acceptable at the moment}
%%%%%%%%%%%%%%%%%%%%%%%%%%%%%%%%%%%%%%%%%

\pagebreak
\section{Compounds} % (fold)
\label{cha:compounds}

\subsection{Chlorinated compounds} % (fold)
\label{sub:chlorinated_compounds}

% subsection chlorinated_compounds (end)

\subsubsection{MCA} % (fold)
\label{ssub:caa}

Calculated and experimental rate constants for the reaction OH $+$ Chloroacetic acid (pKa $=$ 2.866) in anion and acid forms

\input{arrhenuis/MCA-table.tex}
\input{arrhenuis/MCA-fractions.tex}
\input{arrhenuis/MCA-data.tex}
\input{arrhenuis/MCA-split.tex}
\input{arrhenuis/MCA-pure.tex}

\begin{figure}[H]
    \centering
    \includegraphics[width=0.8\textwidth]{arrhenuis/{MCA.double}.pdf}
    \caption{Arrhenius plot for reaction MCA + OH in aqueous phase at different pH values}
    \label{fig:mca}
\end{figure}

% subsection caa (end)

\subsubsection{DCA} % (fold)
\label{ssub:dcaa}

Calculated and experimental rate constants for the reaction OH $+$ Dichloroacetic acid (pKa $=$ 1.301) in anion and acid forms

\input{arrhenuis/DCA-table.tex}
\input{arrhenuis/DCA-fractions.tex}
\input{arrhenuis/DCA-data.tex}
\input{arrhenuis/DCA-split.tex}
\input{arrhenuis/DCA-pure.tex}

\begin{figure}[H]
    \centering
    \includegraphics[width=0.8\textwidth]{arrhenuis/{DCA.double}.pdf}
    \caption{Arrhenius plot for reaction DCA + OH in aqueous phase at different pH values}
    \label{fig:dca}
\end{figure}

% subsubsection dcaa (end)

\subsubsection{TCA} % (fold)
\label{ssub:tcaa}

Trichloroacetic acid \\
Estimation $k \ge 10^6$ M$^{-1}$s$^{-1}$ \\

\input{arrhenuis/TCA-fractions.tex}

% subsubsection tcaa (end)

\subsubsection{2CPA} % (fold)
\label{ssub:2cpa}

Calculated and experimental rate constants for the reaction OH $+$ 2-chloropropionic acid (pKa $=$ 2.96) in anion and acid forms

\input{arrhenuis/2CPA-table.tex}
\input{arrhenuis/2CPA-fractions.tex}
\input{arrhenuis/2CPA-data.tex}
\input{arrhenuis/2CPA-split.tex}
\input{arrhenuis/2CPA-pure.tex}

\begin{figure}[H]
    \centering
    \includegraphics[width=0.8\textwidth]{arrhenuis/{2CPA.double}.pdf}
    \caption{Arrhenius plot for reaction 2CPA + OH in aqueous phase at different pH values}
    \label{fig:2cpa}
\end{figure}

% subsubsection 2cpa (end)

\subsection{Fluorinated compounds} % (fold)
\label{sub:fluorinated_compounds}

% subsection fluorinated_compounds (end)

\subsubsection{MFA} % (fold)
\label{ssub:faa}

Calculated and experimental rate constants for the reaction OH $+$ Fluoroacetic acid (pKa $=$ 2.72) in anion and acid forms

\input{arrhenuis/MFA-table.tex}
\input{arrhenuis/MFA-fractions.tex}
\input{arrhenuis/MFA-data.tex}
\input{arrhenuis/MFA-split.tex}
\input{arrhenuis/MFA-pure.tex}

\begin{figure}[H]
    \centering
    \includegraphics[width=0.8\textwidth]{arrhenuis/{MFA.double}.pdf}
    \caption{Arrhenius plot for reaction MFA + OH in aqueous phase at different pH values}
    \label{fig:mfa}
\end{figure}

% subsubsection faa (end)

\subsubsection{DFA} % (fold)
\label{ssub:dfaa}

Calculated and experimental rate constants for the reaction OH $+$ Difluoroacetic acid(pKa $=$ 1.33) in anion and acid forms

\input{arrhenuis/DFA-table.tex}
\input{arrhenuis/DFA-fractions.tex}
\input{arrhenuis/DFA-data.tex}
\input{arrhenuis/DFA-split.tex}
\input{arrhenuis/DFA-pure.tex}

\begin{figure}[H]
    \centering
    \includegraphics[width=0.8\textwidth]{arrhenuis/{DFA.double}.pdf}
    \caption{Arrhenius plot for reaction DFA + OH in aqueous phase at different pH values}
    \label{fig:dfa}
\end{figure}

% subsubsection dfaa (end)

\subsubsection{TFA} % (fold)
\label{ssub:tfaa}

Trifluoroacetic acid \\
Estimation $k \ge 10^6$ M$^{-1}$s$^{-1}$ \\

\input{arrhenuis/TFA-fractions.tex}

% subsubsection tfaa (end)

\subsubsection{2FPA} % (fold)
\label{ssub:2fpa}

Calculated and experimental rate constants for the reaction OH $+$ 2-fluoropropionic acid (pKa $=$ 2.68) in anion and acid forms

\input{arrhenuis/2FPA-table.tex}
\input{arrhenuis/2FPA-fractions.tex}
\input{arrhenuis/2FPA-data.tex}
\input{arrhenuis/2FPA-split.tex}
\input{arrhenuis/2FPA-pure.tex}

\begin{figure}[H]
    \centering
    \includegraphics[width=0.8\textwidth]{arrhenuis/{2FPA.double}.pdf}
    \caption{Arrhenius plot for reaction 2FPA + OH in aqueous phase at different pH values}
    \label{fig:2fpa}
\end{figure}

% subsubsection 2fpa (end)

\subsubsection{DFPA} % (fold)
\label{ssub:dfpa}

Difluoropropionic acid \\
Estimation $k \ge 10^6$ M$^{-1}$s$^{-1}$ \\

\input{arrhenuis/DFPA-fractions.tex}

% subsubsection dfpa (end)

\subsubsection{TFPA} % (fold)
\label{ssub:tfpa}

Tetrafluoropropionic acid \\
Estimation $k \ge 10^6$ M$^{-1}$s$^{-1}$ \\

\input{arrhenuis/TFPA-fractions.tex}

% subsubsection tfpa (end)

% \subsection{Other} % (fold)
% \label{sub:other}

% % subsection other (end)

% \subsubsection{2Pr-OH} % (fold)
% \label{ssub:2pr_oh}

% NIST: $k = (1.6 \divsymb 2.3) \cdot 10^9$

% % subsubsection 2pr_oh (end)

% section compounds (end)

% \pagebreak
\section{Termodynamic data} % (fold)
\label{sec:termodynamic_data}

\input{arrhenuis/termodyn.tex}

% section termodynamic_data (end)


\end{document}
